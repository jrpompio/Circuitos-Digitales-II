\section{Planes de pruebas}
\subsection{Plan 1}
    Acá se muestra la interacción del modulo ante un deposito.
    Para ello se ingresa la tarjeta, el monto, y el pin correcto,
    como resultado se debe obtener que el balance aumente en la cantidad
    del monto ingresado 
 
    \subsection{Plan 2}
 Acá se muestra la interacción del modulo ante un intento de retiro con
 fondos insuficientes y un retiro efectuado.
 Para ello se ingresa la tarjeta, un monto mayor al balance y el pin correcto
 posteriormente se cambia el valor del monto a un monto menor para poder
 mostrar como se encuente la señal de salida de fondos insuficientes y
 posteriormente esta se apaga y se enciente la señal de entrega de dinero y de
 balance actualizado, debido a que el monto ingresado fue menor al balance.

 \subsection{Plan 3}
 Acá se muestra la interacción del modulo ante el error de pin 3 veces
 mostrando las alartas de advertencia y bloqueo.
Para ello se ingresa la tarjeta, y se ingresa una clave incorrecta
se presiona enter 3 veces y como resultado se obtendrá
que se enciente la salida de pin incorrecto en cada ingreso,
 después la señal de advertencia cuando el contador de intentos llega a 2,
 después la señal de bloqueo cuando el contador de intentos llega a 3.


 \textbf{NOTA:} Todas las pruebas fueron exitosas tanto para el modulo
 conductual como para el modulo sintetizado.

 \section{Instrucciones de simulación}

 Para esta sección se introducirá el archivo README.md que está en la carpeta
 de la tarea 03.

\begin{minted}{python3}
    
### Plan 1:
#### Acá se muestra la interacción del modulo ante un deposito.
### Plan 2:
#### Acá se muestra la interacción del modulo ante un intento de retiro con
#### fondos insuficientes y un retiro efectuado.
### Plan 3:
#### Acá se muestra la interacción del modulo ante el error de pin 3 veces
#### mostrando las alartas de advertencia y bloqueo.



### Copilaciones:
#### Para copilar el plan 1 con el modulo conductual
    make copile
#### Para copilar el plan 2 con el modulo conductual
    make copile2
#### Para copilar el plan 3 con el modulo conductual
    make copile3

#### Para copilar el plan 1 con el modulo sintetizado
    make copile4
#### Para copilar el plan 2 con el modulo sintetizado
    make copile5
#### Para copilar el plan 3 con el modulo sintetizado
    make copile6

### ejecuciones
#### Para ejecutar el script de yosys
    yosys tellermachine.ys
#### Para ejecutar lo copilado
    make run

##### NOTA\: el comando run también elimita el archivo vcd generado y el vpp
##### esto debido a que así se puede ejecutar los archivos recien copilados
##### y así probando que se llega al mismo resultado.

\end{minted}